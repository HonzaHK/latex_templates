\documentclass[11pt,a4paper]{article}
\usepackage[czech]{babel}
\usepackage[utf8x]{inputenc}
\usepackage[T1]{fontenc}
\usepackage[text={17cm,24cm},left=2cm,top=2.5cm]{geometry}
\usepackage{times}
\showboxdepth=\maxdimen
\showboxbreadth=\maxdimen
\providecommand{\uvoz}[1]{\quotedblbase #1\textquotedblleft}
\usepackage{enumitem}

\begin{document}

\noindent \textbf{Dokumentace úlohy:} MKA: Minimalizace konečného automatu v Pythonu do IPP 2014/2015\\
\textbf{Jméno a příjmení:} Jan Kubiš\\
\textbf{Login:} xkubis13\\
\smallskip
\begin{center}
	\begin{LARGE}\textbf{Dokumentace k 2. projektu IPP 14/15}\end{LARGE}\\
	\begin{large}(MKA: Minimalizace konečného automatu)\end{large}
\end{center}
\bigskip

\begin{itemize}[leftmargin=0cm]
\item{\textbf{Zadání úlohy}}
\end{itemize}
Skript \uvoz{mka.py} je implementován v jazyce Python. Slouží pro práci s konečným automatem, především pro jeho minimalizaci. Vstup i výstup jsou v textové formě. Odkud se čte a kam se zapisuje volí uživatel parametry při spouštění.

\begin{itemize}[leftmargin=0cm]
\item{\textbf{Argumenty příkazové řádky a jejich zpracování}}
\end{itemize}
Skript lze spustit jak bez argumentů, tak s některými z následujících:
	\begin{enumerate}
	\item \emph{-\--input=}PATH určuje, kde odkud má skript číst zadaný konečný automat. Při vynechání tohoto parametru se autot čte ze standartního vstupu. Při jeho použití musí být doplněn o validní cestu PATH k souboru. Je-li zadána nevalidní, skript končí s návratovou hodnotou 2.
	
	\item \emph{-\--output=}PATH určuje, kam se bude zapisovat veškerý výstup skriptu. Není-li uveden, proběhne výpis na standartní výstup. Pokud je zadán s nevalidní cestou PATH, skript je ukončen a vrací hodnotu 3.
	\item \emph{-m} zadává se, chceme li automat na vsutpu minimalizovat. Nelze kombinovat s parametrem \emph{-f}.
	\item \emph{-f} automat na vstupu je zpracován, obsahuje-li právě jeden neukončující stav, je tento vypsán jako výstup skriptu. Nelze kombinovat s parametrem \emph{-m}.
	\item \emph{-i} při načítání automatu se nehledí na velikost písma.
	\end{enumerate}

\noindent Nápověda se zobrazí při spuštění s parametrem \emph{--help}.Pokud nebyl zadán přepínač \emph{-m} ani \emph{-f}, na výstupu je vypsán automat, který byl zadaný -- avšak v normalizované formě. Po zkušenostech z předchozího projektu jsem si na argumenty udělal rovnou vlastní parser. Zkontroluje validitu a nastaví řídící proměnné skriptu.


\begin{itemize}[leftmargin=0cm]
\item{\textbf{Zpracování vstupu}}
\end{itemize}
Ze všeho nejdřív je konečný automat na vstupu prohnán naimplementovaným konečným automatem, který z něj odstraní všechny komentáře, a taky nahradí řídící znaky pro popis automatu (obyčejné závorky, složené závorky, čárka, apostrof, znak začátku komentáře) vlastním textem -- ty se totiž mohou objevit i jako symbol vstupní abecedy, a proto je potřeba je rozlišit. Řešit tento problém regulárními výrazy mi přišlo zdlouhavé. Stačilo zavést automat, který se řídí pomocí boolean proměnné držící informaci o tom, zda-li se nacházíme ve stavu zapisování symbolu vstupní abecedy nebo mimo něj. To je asi vše co se dá v této chvíli dělat -- následuje kontrola lexikální a syntaktické validity zápisu automatu pomocí regulárních výrazů. Například odstranění případných mezer není nyní možné (\emph{Q=\{s1,s\textvisiblespace 2,s3\}} je chybný zápis, avšak \emph{Q=\{s1,s2,s3\}} už správný). Celek je reguláry rozdělen a každá komponenta pětice nahrána zvlášť. V dalším kroce probíhá jejich kontrola. Položky komponenty jsou nahrány do seznamu jejím rozdělením v místě řídící čárky. Vypuštění všech bílých znaků může proběhnout po kontrole jednotlivých položek -- regulární výrazy provádějící tuto kontrolu tedy musí být doplněny o možné bílé znaky na správných místech. Prázdný řetězec není symbol, pokud se tedy objeví ve vstupní abecedě, chyba se hlásí odtud. Takto probíhá \textbf{lexikální a syntaktická kontrola} zadaného automatu.
\newpage
\bigskip
Po dokončení předchozí části proběhne \textbf{sémantická kontrola}. Položky všech komponent jsou uloženy v poli v normalizovaném tvaru, můžou se tedy vzájemně porovnávat. Konkrétně počáteční stav, koncové stavy a části položek přechodových pravidel jsou kontrolovány, zda-li neobsahují stav resp. symbol který nepatří do množiny všech stavů resp. do vstupní abecedy. Pokud ano, skript hlásí sémantickou chybu. Stejnětak je-li množina reprezentující vstupní abecedu prázdná.

Po dokončení předchozí části proběhne \textbf{kontrola na dobrou specifikovanost} automatu. Je vyžadováno, aby pro každý stav existovalo právě jedno (determinismus) přechodové pravidlo pro každý symbol. Kontroluje se tedy počet pravidel se součinem kardinalit zmíněných množin a také shodnost levých stran pravidel. Je-li v~pravidlech uveden epsilon přechod, chyba se hlásí odtud. Nakonec je potřeba testovat existenci nedostupných a~neukončujících stavů (při nálezu nedostupného stavu je hlášena chyba, neukončující stav může být maximálně jeden -- ten se zde i případně nahraje do speciální proměnné, se kterou se pracuje, byl-li zadán argument~\emph{-f}).


\begin{itemize}[leftmargin=0cm]
\item{\textbf{Tvorba výstupu}}
\end{itemize}
V této chvíli máme načtený dobře specifikovaný automiat. Výstup záleží na zvoleném přepínači \emph{-m} /\emph{-f}. Pokud uživatel nezvolil ani jeden, obsah komponent automatu je lexigofraficky uspořádán, poté podle pravidel normalizovaného výstupu poskládán do výstupního řetězce. Ten je následně vypsán na výstup zvolený argumentem \emph{-\--output}. Volil-li argument \emph{-m}, ještě před uspořádáním a výpisem je automat minimalizován. V případě argumentu \emph{-f} se pouze vypíše nalezený neukončující stav (případně 0) a skript končí.


\begin{itemize}[leftmargin=0cm]
\item{\textbf{Minimalizace}}
\end{itemize}
Algoritmus minimalizace z přednášek je na pochopení poměrně jednoduchý. Implementace však už tak jednoduchá nebyla. Štěpení množiny stavů se mi nepodařilo zajistit jinak než spoustou zanořených cyklů -- vzhledem k tomu, že se potřebují prostřídat všechny možné kombinace prvků pěti množin to nejspíš ani jinak možné není. Za největší překážku považuju problém rozdělení množin podle stavu na pravé straně vybraných pravidel. V ukázkovém řešení minimalizace v předmětu IFJ je totiž popsán ideální případ, kdy se množiny buď vůbec nedělí a nebo se dělí na dvě další. Při jiném zadání by se ale mohlo stát, že se bude rozdělovat po X prvcích na Y množin. Po dlouhém přemýšlení ale vyšlo najevo, že toto není potřeba řešit -- stačilo vždy příslušnou množinu rozdělit ne na Y, ale na 2 množiny. Zbylá množina stavů je totiž rozšštěpena (je-li to možné) hned v dalším průchodu (zajišťuje jej cyklus \emph{while}, který pokračuje, dokud se množina mění -- je co rozdělovat). Před dalším průchodem jsou nalezené prvky odebrány z míst v původní množině ke štěpení a jsou přidány jako další samostatná množina. Pokud je štěpení hotové, podle výsledků jsou přetvořeny komponenty KA -- množina stavů, pravidel, počáteční stav a množina koncových stavů.

\end{document}

