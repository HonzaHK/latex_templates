\documentclass[12pt,a4paper,titlepage]{extarticle}
\usepackage[czech]{babel}
\usepackage[utf8x]{inputenc}
\usepackage[T1]{fontenc}
\usepackage[text={17cm,24cm},left=2cm,top=2.5cm]{geometry}
\usepackage{times}
\usepackage{enumitem}
\showboxdepth=\maxdimen
\showboxbreadth=\maxdimen
\providecommand{\uvoz}[1]{\quotedblbase #1\textquotedblleft}
\usepackage{enumitem}

\begin{document}

\begin{titlepage}
\begin{center}

\begin{Huge}\textsc{Fakulta informačních technologií\\Vysoké učení technické v Brně}\end{Huge}

\vspace{\stretch{0.382}}
\begin{LARGE} \textbf{ISA projekt\,--\,Dokumentace}\end{LARGE}
\\ \vspace{2mm}
\begin{LARGE}Offline NetFlow sonda\end{LARGE}
\vspace{\stretch{0.618}}

\end{center}

\begin{large}\today  \hfill  Jan Kubiš\end{large}
\end{titlepage}

\tableofcontents
\newpage

\section{Uvedení do problematiky}

\subsection{Zadání úlohy}
Cílem je implementace analyzátoru zachycené síťové komunikace ve formatu PCAP, který bude fungovat jako offline NetFlow sonda. Musí být také schopen exportovat vypočítané statistiky na kolektor.

\subsection{NetFlow}
Technologie NetFlow byla vyvinuta společností Cisco systems. Slouží především k~monitorování toku (flow) dat v~síti (odkud, kam, typ spojení, délka spojení, velikost dat~apod.). Výsledky tedy mohou po správné interpretaci významně pomoci s~pochopením pochodů uvnitř sítě a~tedy s~její případnou optimalizací, zabezpečením, rozšiřováním či jakoukoliv jinou modifikací.
\\Celý proces běžně zajišťují 3~komponenty:
\begin{itemize}[noitemsep,nolistsep]
\item exporter -- packet sniffing, zařazování paketů do příslušných flows, kontrola expirace flows, výpočet statistik a jejich export na kolektor (cíl projektu)
\item kolektor -- agreguje a uchovává data z jednoho až několika exporterů
\item interpret -- vhodně interpretuje nasbíraná data
\end{itemize}
Existuje více NetFlow verzí, tento projekt pracuje s~verzí NetFlow v5.

\subsection{Pasivní NetFlow sonda}
Pasivní NetFlow sonda je alternativní architektura k~nepříliš efektivní klasické architektuře, kdy je exportér umístěn přímo na routeru, a~zpomaluje tak jeho směrovací činnost. To lze kompenzovat například diskrétním vzorkováním, avšak za cenu snížení přesnosti statistik, které v~nejhorším případě nemusí být nijak vypovídající.
\\Proto vznikl přístup pasivních sond, které směrovačů této zátěže zbavují. Dedikované zařízení je v~místě potřeby transparentně připojené na linku a~odposlouchává všechny průchozí pakety. Z~těch poté vypočítá a~exportuje statistiky na kolektor.

\subsection{NetFlow v5}
Verze NetFlow specifikuje, která data z~paketů (případně které pakety) budou zahrnuty do statistik, a~jak budou jednotlivé položky statistik uspořádány při exportu na kolektor\,--\,tzv.~NetFlow~diagram. Ten se skládá z~headeru a~recordu. Co která část obsahuje je možno dohledat na stránkách Cisco Systems.
\\Pro naše účely bude implementována verze~5, která je v~současnosti nejrozšířenější. Zpracovává pouze pakety se síťovým protokolem IPv4. Statistiky jsou na kolektor přenášeny v~UDP paketu ve formátu header+n*record, kde n~nabývá maximálně hodnoty 30.

\pagebreak

\section{Návrh aplikace}

\subsection{Offline sonda}
Vytvářená aplikace je sonda typu offline. To znamená, že analyzovaná komunikace nebude probíhat v reálném čase přímo na síťové kartě, nicméně bude načtena ze souboru, případně ze STDIN. Veškeré časové údaje nutné pro řízení skriptu nebo výpočet statistik budou čerpány z jednotlivých paketů. Čas tedy bude plynout diskrétně, od hodnoty k hodnotě, na základě timestampu právě zpracovávaného paketu. Z toho vyplývá, že po zpracování posledního paketu ze souboru bude potřeba uměle skočit v čase do nekonečna (pouze hypoteticky, toto bude řešeno implementací vhodného workaroundu), jelikož by se čas exporteru zastavil na jeho čase a dál nepokračoval, na rozdíl od real-time sondy.

\subsection{Explicitní specifika}
Na rozdíl od klasických pravidel agregace bude v~tomto projektu každý UDP paket tvořit samostatnou flow o~nulové délce trvání (okamžitá expirace). TCP spojení budou sledována pouze jednosměrně. To má dopad hlavně na uzavírání spojení: pošle-li jedna strana paket s~příznakem FIN nebo RST, flow do které náleží bude okamžitě expirována. Exporter dále zpracovává pakety ICMP a~IGMP (okamžitá expirace v obou případech).

\subsection{Řídící argumenty}
Skript lze spustit jak bez argumentů, tak s některými z následujících:
	\begin{itemize}[noitemsep,nolistsep]
	\item \emph{-i /-\--input <PATH>} určuje, odkud bude aplikace načítat zachycenou komunikaci. Při vynechání tohoto parametru je komunikace očekávána na standartním vstupu. Při jeho použití musí být doplněn o validní cestu \emph{PATH} k~souboru. Je-li zadána nevalidní, skript vypíše chybu, stejně tak v~případě špatného formátu vstupního souboru.	
	\item \emph{-c -\--collector <IP|hostname:port>} určuje adresu na kterou se budou zasílat vypočítané statistky. Volitelně lze zadat i~port cílového zařízení. Implicitně~localhost:2055.
	\item \emph{-I -\--interval <sec>} udává interval, v~jakém bude sonda provádět export. Implicitně~300.
	\item \emph{-m -\--max-flows <cnt>} maximální držitelý počet aktivních flows. V~případě, že by měl být překročen, vyexpiruje se flow s~nejdelším timeoutem a~tak uvolní místo nově přidávané. Implicitně~50.
	\item \emph{-t -\--tcp-timeout <sec>} pokud některá z~TCP flows bude neaktivní déle než zadaná hodnota, bude vyexpirována. Implicitně~300. 
	\end{itemize}
\noindent Nápověda se zobrazí při spuštění s parametrem \emph{--help}.\\Při spuštění s parametrem \emph{-b} se vypíše očekávaný počet bodů za projekt.

\subsection{Pořadí operací}
Real-time sonda tedy umí v~jakémkoliv bodě zjistit aktuální čas (vyprší nastavený časovač) a~je-li to nutné, vykonat požadované operace. Offline sonda musí s~každým novým paketem ze všeho nejdříve zjistit, zda neuplynul interval pro export a~zda některé z aktivních flows nemají být expirovány na základě tcp-timeout. Teprve potom mohou daný paket zpracovat a~případně zařadit do flow, nikoliv v~opačném pořadí! Po rozhodnutí, že daný paket nepatří do žádné z~aktivních flows, ale založí novou, je potřeba porovnat počet všech aktivních s maximálním možným počtem.

\pagebreak

\section{Popis Implementace}
Na doporučení kolegů je program implementován v~jazyce Python, ve kterém je dostupná velmi vhodná knihovna Scapy. Verze předinstalovaná na virtuálním disku má v~sobě určité chyby, napřiklad počítá špatně velikost DNS paketů, proto je potřeba ji nejprve aktualizovat na nejnovější verzi 2.3.1, což se děje pomocí Makefile. Ten také udělí potřebná práva pro spušťení skriptu.

\subsection{Zpracování argumentů příkazové řádky}
Z~předchozích zkušeností pro parsování argumentů nepoužívám knihovní funkci getopt a~proto byla implementována funkce přizpůsobená speciálně tomuto projektu, která je mimo jiné nachystána na případné modifikace. Kontroluje se například duplicita zadaných argumentů, validita zadané IP adresy a~hodnoty portu, kontrola existence zadaného vstupního souboru a~jeho formátu.

\subsection{Hlavní smyčka}
Většina běhu programu se odehrává ve for cyklu, který iteruje přes všechny pakety ze souboru. Při každém dalším paketu se aktualizuje čas sondy. V~zápětí probíha kontrola ohledně vypršení intervalu pro export. V~dalším kroku je zpracovávaný paket předán do funkce, která pokračuje pouze za předpokladu, že je typu IPv4 s~protokolem TCP/UDP/ICMP/IGMP. Tato funkce hledá flow, do které by mohl být zařazen. Pro ušetření výpočetního času v tomto kroce zároveň probíhá kontrola timeoutu všech aktivních flows. Ta je samozřejmě testována ještě před zařazením paketu do flow, a~pokud by měla být expirována, přeskakuje se (do té už aktuální paket zařadit nemůžeme). Flow je implementována jako třída s~příslušnými atributy, v~rámci ní jsou zde dvě statická pole: jedno pro aktivní a~druhé pro expriované flows. Pokud byla nalezena odpovídající flow, je do ní paket přidán, a~to pomocí funkce, ktera na základě něj modifikuje dosavadní statistiky pro tuto flow (velikost, timestamp, počet paketů..), případně ji vyexpiruje (TCP s flagem FIN/RST). Pokud nalezena nebyla, je vytvořena nová instance třídy Flow, která při svém vytváření pomocí \emph{init} také rozhodne, do kterého ze dvou polí patří (může jít rovnou do expirovaných - UDP/ICMP/IGMP nebo TCP s~flagem FIN/RST). Po jejím vytvoření se kontroluje, zda nebyl přesažen maximální počet aktivních flows. Pokud ano, pomocí for cyklu se ze všech aktivních flows vybere ta, která byla nejdéle neaktivní a~expiruje se.

\subsection{Exportace}
Z~definice NetFlow v5~diagramu vyplývá, že se může exportovat maximálně po 30 záznamech. Funkce pro export tedy obsahuje řídící while, který se opakuje, dokud je pole expirovaných neprázdné. Je-li počet zbývajících záznamů větší než 30, vytvoří se dočasné pole obsahující prvních 30, které se z~původního smažou. Je-li menší, do pole se přesune zbytek. Tuto práci zajišťuje slice metoda. Toto dočasné pole je při každém průchodu posláno do funkce k sestavení v5~diagramu, který je funkcí vracen. Některé položky headeru a~recordu nejsou z~komunikace zjistitelné, do těch je natvrdo vložena nekonfliktní hodnota. Vrácený diagram je poté poslán do funkce k~odeslání diagramu. Odesílá se přes UDP paket na adresu specifikovanou v argumentech při spouštění.
\pagebreak

\section{Informace o programu}
\begin{itemize}[noitemsep]
	\item Implementační jazyk: Python
	\item Počet souborů: 1
	\item Počet řádků: 388
	\item Velikost souboru: 11.1 kB
\end{itemize}

\section{Návod na použití}
\begin{itemize}[noitemsep]
	\item příkazem \emph{make} se provede stažení a~instalace/aktualizace podpůrných knihoven a~nastavení potřebných práv pro spuštění skriptu
	\item skript spouštějte s~následující syntaxí: \\./isaexporter [-i <file>] [-c <ip|hostname>[:<port>]] [-I <sec>] [-m <cnt>] [-t <sec>]
\end{itemize}

\end{document}


